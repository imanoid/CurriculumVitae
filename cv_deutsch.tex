%----------------------------------------------------------------------------------------
%	CURRICULUM VITAE
%----------------------------------------------------------------------------------------

\makecvtitle % Print the CV title

%----------------------------------------------------------------------------------------
%	EDUCATION SECTION
%----------------------------------------------------------------------------------------

\section{Ausbildung}

\cventry{2014--2016}{Master of Science - Multimodal and Cognitive Systems}{Universität Zürich}{}{}{Beinhaltet folgende Module:
\begin{itemize}
	\item Fundamentals of Image Processing and Computer Vision
	\item Autonomous Mobile Robots
	\item Introduction to Machine Learning (ETH Course)
	\item Neuromorphic Engineering II
	\item Neurophysics
	\item Complex Systems: Berechenbares Chaos in dynamischen Systemen
	\item Master Project: Head Pose Tracking with Quadrotors with Headmounted camera
\end{itemize}}  % Arguments not required can be left empty

\cventry{2009--2013}{Bachelor of Science - Angewandte Informatik: Neuroinformatics}{Universität Zürich}{}{}{Beinhaltet folgende Module:
\begin{itemize}
	\item Neuromorphic Engineering I
	\item Theory, Programming and Simulation of Neural Networks
	\item Systems Neuroscience
	\item Computational Vision
	\item Introduction to Neuroinformatics
\end{itemize}}

\cventry{2008--2009}{Passerelle}{AKAD College}{}{}{Die Passerelle ist eine Prüfung, welche dem Besitzer, zusammen mit einer Berufsmatura, zur Hochschulreife verhilft.}

\cventry{2003--2007}{Informatikmittelschule}{Kantonsschule Enge}{}{}{
Die Informatikmittelschule ist eine Ausbildung als Applikationsentwickler, ähnlich einer Lehre. Die Ausbildung besteht aus drei Jahren Schule, welche die üblichen Schulfächer abdeckt mit einem Fokus auf Betriebsökonomie. Das vierte und letzte Jahr ist ein Praktikum, währenddessen man eine zweiwöchige IPA (Individuelle Praktische Arbeit) absolvieren muss. Mein Projekt war ein Prototyp eines chemischen Monitoring Systems für Kläranlagen.}

\cventry{1999--2003}{Gymnasium, Altsprachliches Profil}{Kantonsschule Freudenberg}{}{}{Ich habe 4 Jahre Gymnasium mit einem Fokus auf Sprachen (Deutsch, Französisch, Spanisch, Latein) absolviert und dann zu Gunsten einer Ausbildung an der Informatikmittelschule abgebrochen.}

\section{Master Arbeit}

\cvitem{Title}{\emph{Paper Tracking} - A real time algorithm}
\cvitem{Supervisor}{Prof. Dr. Chat Wacharamanotham}
\cvitem{Description}{Ich habe einen Algorithmus entwickelt, welcher dazu in der Lage ist ein Paper in einem Bild zu finden, das von einer Kamera aufgenommen wurde, welche auf der Stirn des Lesers Montiert ist. Der Algorithmus sucht nach Bereichen in denen eine hohe Dichte von hohen Frequenzen gefunden werden. Das so gefundene Paper wird dann gegen eine PDF Datenbank gematcht mit Hilfe von SURF Features. Zuletzt wird dann die gematchte PDF Seite nach statistischen Grafiken abgesucht unter verwendung von Visual Words basierten Feature Vektoren und einer SVM mit einem RBF Kernel.}
\cvitem{Grade}{\textbf{5.75}}

\section{Bachelor Arbeit}

\cvitem{Title}{\emph{Maggie} - An Infant Robot Model}
\cvitem{Supervisor}{Dr. Hugo Gravato Marques}
\cvitem{Description}{I built a physical prototype of an infant robot model including an API to control it from a PC and collect data. It is made out of spare metal parts, servo motors, force resisting sensors, two cameras, IMUs and 3D designed and printed parts.}
\cvitem{Grade}{\textbf{5.5}}

%----------------------------------------------------------------------------------------
%	WORK EXPERIENCE SECTION
%----------------------------------------------------------------------------------------

\section{Erfahrung}

\cventry{2012--2015}{Software Developer}{\textsc{New Voice AG}}{Zürich}{}{New Voice AG is a company located in Zürich which develops a software called MobiCall which is an Alarm-, Information-, Conference and Callrecording system which is built on top of a PBX system. My responsibility was to develop and maintain a plugin for locating Dect and WiFi devices inside buildings and triggering alarms.}

%------------------------------------------------

\cventry{2011--2012}{Software Developer}{\textsc{TrueBPM Solutions AG}}{Zürich}{}{TrueBPM develops a BPM system that is used internally by an assurance company. While I was working there, I was responsible for letter generation. I had to design BIRT reports based on word documents and also do some server backend programming. I left the company after a year, because the project was finished and I wanted to do something more challenging.}

%------------------------------------------------

\cventry{2006--2011}{Software Developer}{\textsc{IT-GR GmbH}}{Zürich}{}{IT-Gr GmbH is a software developing company. During my time there I worked alone on a monitoring system for water recycling plants for a customer called Unimon GmbH. The software periodically pulls measured data from different sensors embedded in the water plant. Users can log in through a web interface, analyze the data and create reports. It is also possible to define conditions under which alarms are  triggered. It then automatically contacts the responsible person per E-Mail or SMS.
\newline
My first year there (2006-2007) was an Internship as part of the “Informatikmittelschule”.}

%----------------------------------------------------------------------------------------

%----------------------------------------------------------------------------------------
%	REFERENCE CONTACTS SECTION
%----------------------------------------------------------------------------------------

\newpage

\section{Referenzpersonen}

\subsection{Prof. Dr. Chat Wacharamanotham}
\cvitem{Link}{\href{http://www.ifi.uzh.ch/en/zpac/people/chat.html}{http://www.ifi.uzh.ch/en/zpac/people/chat.html}}
\cvitem{}{Supervision of Master Thesis}
\cvitem{E-Mail}{\href{mailto:chat@ifi.uzh.ch}{chat@ifi.uzh.ch}}
\cvitem{Telephon}{+41 (0)44 635 43 13}

\subsection{Prof. Dr. Davide Scaramuzza}
\cvitem{Link}{\href{http://rpg.ifi.uzh.ch/people\_scaramuzza.html}{http://rpg.ifi.uzh.ch/people\_scaramuzza.html}}
\cvitem{}{Supervision of Master Project}
\cvitem{E-Mail}{\href{mailto:sdavide@ifi.uzh.ch}{sdavide@ifi.uzh.ch}}
\cvitem{Phone}{+41 (0)44 635 24 09}

%----------------------------------------------------------------------------------------
%	COMPUTER SKILLS SECTION
%----------------------------------------------------------------------------------------

\section{Computer skills}

\cvitem{Basic}{\textsc{Bash Scripts}, QT, Cuda}
\cvitem{Intermediate}{\textsc{java}, \textsc{html}, \textsc{JavaScript}, \LaTeX, Microsoft Windows, OpenCV, Computer Hardware, Scikit Learn}
\cvitem{Advanced}{\textsc{python}, \textsc{C++}, Linux}

%----------------------------------------------------------------------------------------
%	LANGUAGES SECTION
%----------------------------------------------------------------------------------------

\section{Languages}

\cvitemwithcomment{German}{Mothertongue}{}
\cvitemwithcomment{Spanish}{Mothertongue}{}
\cvitemwithcomment{English}{Intermediate}{Conversationally fluent}

%----------------------------------------------------------------------------------------

%----------------------------------------------------------------------------------------
%	INTERESTS SECTION
%----------------------------------------------------------------------------------------

\section{Interests}

\renewcommand{\listitemsymbol}{-~} % Changes the symbol used for lists

\cvlistdoubleitem{Reading}{Jiu Jitsu}
\cvlistdoubleitem{Karate}{Cooking}
\cvlistdoubleitem{Workout}{Swimming}

\newgeometry{bottom=0pt}
\section{Master Grades - Exempt from module booking system}
\includegraphics[width=\textwidth]{pictures/master/module1.png}
\includegraphics[width=\textwidth]{pictures/master/module2.png}

\enlargethispage{12pt}
\newpage
\section{Bachelor's Diploma}
\includegraphics[width=\textwidth]{pictures/bachelor/page0.png}
\includegraphics[width=\textwidth]{pictures/bachelor/page1.png}
\includegraphics[width=\textwidth]{pictures/bachelor/page2.png}
\includegraphics[width=\textwidth]{pictures/bachelor/page3.png}

\restoregeometry